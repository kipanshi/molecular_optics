\newpage
\vskip 10mm
\noindent{\Large Предисловие}\vskip 3mm 
\vskip 1cm
%\nglav 0\global\advance\nglav by 1 
\markboth{\hfill\small Основы молекулярной оптики \hfill}{}
\markright{\hfill\small Предисловие\hfill}{}

\pagestyle{empty}

\thispagestyle{empty}


Внутренняя уверенность в собственной правоте и спокойствие приходят к кому-то с годами, после терзаний и треволнений, ниспосланных свыше. Кто-то так и не обретает этого чувства гармоничного созерцания жизни и собственного влияния на нее. А некоторым дано от природы аристократично, с достоинством и величием, не только быть очевидцем собственного существования, а бороться за справедливость, вершить свою судьбу, родных, близких, помогать окружающим, задавать вектор развития в различных областях жизнедеятельности (от науки и до каких-то бытовых вещей), оставаться эталоном честного отношения ко всему и добродетельного по отношению ко всем людям.

Не знаю почему, но написать это предисловие к научной книге Натальи Борисовны Рождественской мне захотелось не где-нибудь, а в самолете, находясь на высоте одиннадцати тысяч метров над землей. Образно говоря --- это лишь малая часть дистанции, которую еще предстоит преодолеть, дабы дотянуться до планки, установленной этим уникальным человеком, чья жизнь была и дерзанием, и подвигом, и служением, и любовным романом, и трагедией одновременно.

Она родилась 4 мая 1931 года в семье капитана 1-ого ранга Бориса Петровича Гаврилова и графини Евгении Александровны фон Патон. Ее бабушка (немка) появилась на свет в Нарве, а дед --- капитан 1-ого ранга, строил в Германии крейсер <<Новик>>, прадед --- немецкий барон, прабабушка --- урожденная баронесса Арпсгоффен. Понятно, что с такой не рабоче-крестьянской биографией и воспитанием Наталье Борисовне Рождественской и членам ее семьи были гарантированы и обеспечены многочисле                                                       нные испытания с проблемами в <<стране Cоветов>>. Гонениям со стороны НКВД подвергались все совершеннолетние Рождественские. Их дети на себе ощущали, что такое классовая ненависть.

Тем не менее, молодая и не здешне красивая (словно сошедшая с картин Ренуара) Наталья Рождественская поступила в Санкт-Петербургский Университет на факультет физики. Выпустилась в 1954 году с отличием. И с тех пор, до последнего дня, она читала лекции на географическом , экономическом и биолого-почвенном факультетах. Каждый курс по «Молекулярной оптике» и «Физике конденсированного состояния» для новых студентов она начинала рассказом о тех замечательных и без преувеличения --- великих ученых, ставших основоположниками направления. Она сперва рассказывала об их научных открытиях и достижениях, а потом вспоминала, как академики выбрасывались в лестничные пролеты с пятого этажа в момент ареста или занимались наукой в небезызвестных сталинских <<шарашках>>.

Аристократическая любовь к шампанскому <<Брют>>\ на завтрак и всегда по этикету сервированный стол с яствами из каких-то кухмистерских учебников, хоть и сопровождали Наталью Рождественскую по жизни, но они лишь были приложением к ее тактичной манере общения, коей она всячески радовала собеседников. Зачастую казалось, что именно эти изысканные церемониалы приема гостей, позволяли ей справляться с тяготами преследований и избавляли от одиночества, заглушали боль утрат.

\thispagestyle{myheadings}

Ярче и лучше всего о масштабе личности человека свидетельствуют его дети. Дарованное родителями домашнее воспитание и образование, в сочетании с полученными Натальей Рождественской академическими познаниями --- все это передалось ее сыну Дмитрию Рождественскому (основоположнику современных телевизионных и интернет-технологий, создателю первого в СССР независимого от государства телевидения --- телекомпании <<Русское видео>>). Наталья Борисовна развила в нем творческие таланты, креативность, образность мышления и внутреннюю неуспокоенность достигнутым.

\begin{figure}[tbp]
\centerline{\hbox{\includegraphics[scale=1.3]{Ris/ris_eps/babushka_bw.eps}}}
\end{figure}

Свободомыслие в несвободной стране конечно же не могло не сказаться на жизни Натальи Рождественской. Долгие годы была <<невыездной>>\ из СССР, ей не давали учавствовать в  международных научных обменах. Плюс к этому, советские партийные и чекистские руководители знали о многочисленных зарубежных родственниках госпожи Рождественской (баронах, герцогах и графах), что тоже влекло за собой пристальное внимание органов и запрет на общение с зарубежьем. Но невзирая на преграды и противодействие, более 90 публикаций профессора Натальи Рождественской появились в российских и иностранных журналах, неоднократно она выступала с докладами на международных конференциях и была широко признана как специалист в области рассеяния света в жидкостях, растворах и жидких кристаллах. Она оказалась одним из лучших учеников и продолжателей дела величайших физиков М.В. Волькенштейна, У.Ф. Гросса, М.Ф. Вукса, Когда же <<железный занавес>>\ пал, Наталья Рождественская работала в Европейском центре молекулярной оптики в Бордо, стала членом EMLG (Европейская группа молекулярных жидкостей). Ее авторское свидетельство позволило создать термодинамический эталон чистой воды, необходимый для обеспечения достоверности полученных экспериментальных результатов при изучении свойств воды и водных систем, при решении проблем физики, молекулярной биофизики и физической химии.

Разбившийся в дребезги Советский Союз одним из своих осколков все же задел и, практически, смертельно ранил Наталью Рождественскую уже на излете 90-ых годов двадцатого века. Тогда нереформированная сталинская репрессивная система опять постучалась в ее дом и, по сфабрикованному политическому уголовному делу, был арестован ее сын Дмитрий Рождественский. Два года его продержали в следственных изоляторах страны, а Наталья Рождественская каждый день ездила с передачами по тюрьмам, с заявлениями по судам и с требованиями по прокуратурам и ФСБ. Наталья Рождественская находила в себе силы читать лекции в университет и рассказывать о рассеянии света, о датчиках определения барометрической высоты, о приборах контроля высокодисперсных примесей в деионизованной воде для предприятий электронной промышленности.

Не добились следователи от сына Натальи Рождественской оговора других людей и ложных показаний против первого мэра Санкт-Петербурга Анатолия Собчака. Его отпустили домой, больного и измотанного ночными обысками, допросами, судами и бесчеловечными условиями содержания. Здоровый дух оказался у Дмитрия сильнее физического здоровья организма и спустя год после освобождения сын Натальи Рождественской умер из-за перенесенного инсульта.

\thispagestyle{myheadings}

После гибели сына она живет памятью о нем, воспитывает внучку Анастасию, вкладывает в нее все те черты характера и знания, за которые окружающие так ценят Наталью Рождественскую и ее сына Дмитрия. Она пишет эту научную работу, уединяясь на веранде второго этажа своей фамильной дачи в Сиверской. Для нее уже ничего не существует кроме внучки (которая уже почти стала самостоятельной), науки, общения со студентами, воспоминаний…. Когда она уже завершала работу над рукописью этой книги, Рабы Божьей Натальи Борисовны Рождественской, урожденной графини Фон Патон не стало. А нам теперь дарованы ее научные работы, знания, открытия, достижения и трудно досягаемая высота, к которой не сможет поднять ни один самолет, к тем космическим далям, где орбитальные корабли через годы будут осуществлять вычисленную и разработанную Натальей Рождественской мягкую посадку на Марс.


\vskip 4mm
\hfill{\it Руслан Линьков}

\pagestyle{myheadings}
