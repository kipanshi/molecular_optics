\ \lit\snum 0
\col{����������}\vskip 3mm
{\rmel
\parindent 8mm

\snitem {\itel V. I. Tikhonov and A. A. Volkov}, Separation of
water into its ortho and para isomers, Science 296 (2002) 2363.%1

\snitem {\itel R. B. Martin}, Localized and spectroscopic
orbitals: Squirrel ears on water, J. Chem. Ed. 65 (1988) 668-670.%2

\snitem {\itel M. Laing}, No rabbit ears on water, J. Chem. Ed. 64
(1987) 124-128.%3

\snitem {\itel C. W. Kern, M. Karplus,} The water molecule, in F.
Franks (Ed), Water A comprehensive treatise, Vol. 1, (Plenum
Press, New York, 1972) pp. 21-91.%4

\snitem {\itel J. B. Hasted,} Liquid water: Dielectric properties,
in F. Franks (Ed), Water A comprehensive treatise, Vol 1, (Plenum
Press, New York, 1972) pp. 255-309.%5

\snitem {\itel P. L. Silvestrelli and M. Parrinello,} Structural,
electronic, and bonding properties of liquid water from first
principles, J. Chem. Phys., 111 (1999) 3572-3580.%6

\snitem {\itel K. Ichikawa, Y. Kameda, T. Yamaguchi, H. Wakita and
M. Misawa,} Neutron-diffraction investigation of the
intramolecular structure of a water molecule in the liquid-phase
at high-temperatures, Mol. Phys. 73 (1991) 79-86.%7

\snitem {\itel F. Franks,} Water: 2nd Edition A matrix of life,
(Royal Society of Chemistry, Cambridge, 2000).%8

\snitem {\itel J. A. Odutola and T. R. Dyke,} Partially deuterated
water dimers: Microwave spectra and structure, J. Chem. Phys. 72
(1980) 5062-5070.%9

\snitem {\itel J. L. Finney}, The water molecule and its
interactions: the interaction between theory, modelling and
experiment, J. Mol.
Liquids, 90 (2001) 303-312.%10

\snitem {\itel P. F. Bernath}, The spectroscopy of water vapour:
Experiment, theory and applications, Phys. Chem. Chem. Phys. 4
(2002)
1501-1509.%11

\snitem {\itel R. Lemus}, Vibrational excitations in H2O in the
framework of a local model, J. Mol. Spectrosc. 225 (2004) 73-92.
{\itel A. Janca, K. Tereszchuk, P. F. Bernath, N. F. Zobov, S. V.
Shirin, O. L. Polyansky and J. Tennyson,} Emission spectrum of hot
HDO below 4000
cm$^{-1}$, J. Mol. Spectrosc. 219 (2003) 132-135.%12

\snitem {\itel D. Eisenberg and W. Kauzmann}, The structure and
properties of
water (Oxford University Press, London, 1969).%13

\snitem {\itel H. R. Zelsmann}, Temperature dependence of the
optical constants for liquid H2O and D2O in the far IR region, J.
Mol. Struct. 350
(1995) 95-114.%14

\snitem {\itel J. K. Vij, D. R. J. Simpson and O. E. Panarina,}
Far infrared spectroscopy of water at different temperatures: GHz
to THz dielectric spectroscopy of water, J. Mol. Liquids 112 (2003
)
125-135. %15

\snitem {\itel V. I. Gaiduk and J. K. Vij}, The concept of two
stochastic processes in liquid water and analytical theory of the
complex permittivity in the range 0 - 1000 cm-1,
CPS:physchem/01.09.2005.
uploaded 11 Sept. 2001.%16

\snitem {\itel K. Johnson}, "Water buckyballs" Chemical, catalytic and cosmic
implications, Infinite Energy 6 (2000) 29-32. {\itel K. H. Johnson and B.
Zhang}, Stabilized water nanocluster-fuel emulsions designed
through quantum chemistry, United States Patent 5,997,590 (1999).%17

\snitem {\itel J. A. Padro and J. Marti}, An interpretation of the low-frequency
spectrum of liquid water, J. Chem. Phys. 118 (2003) 452-453.%18

\snitem {\itel K. N. Woods and H. Wiedemann}, The relationship between dynamics
and structure in the far infrared absorption spectrum of liquid
water, Chem. Phys. Lett. 393 (2004) 159-165. %19

\snitem {\itel J. D. Worley and I. M. Klotz}, Near-infrared spectra of H2O-D2O
solutions, J. Chem. Phys. 45 (1966) 2868-2871. {\itel G. E. Walrafen, M.
S. Hokmabadi and W. H. Yang}, Raman isobestic points from liquid
water, J. Chem. Phys. 85 (1986) 6964-6969.%20

\snitem {\itel A. A. Yakovenko, V. A. Yashin, A. E. Kovalev and E. E. Fesenko},
Structure of the vibrational absorption spectra of water in the
visible region, Biophysics 47 (2002) 965-969.%21

\snitem {\itel S. Woutersen, U. Emmerichs and H. J. Bakker,} Femtosecond mid-IR
pump-probe spectroscopy of liquid water: evidence for a
two-component structure, Science 278 (1997) 658-660.%22

\snitem {\itel M. F. Kropman and H. J. Bakker}, Dynamics of water molecules in
aqueous solvation shells, Science 291 (2001) 2118-2120.%23

\snitem {\itel M. Starzak and M. Mathlouthi,} Cluster composition of liquid water
derived from laser-Raman spectra and molecular simulation data,
Food Chem. 82 (2003) 3-22.%24

\snitem {\itel C. L. Braun and S. N. Smirnov}, Why is water blue, J. Chem. Edu. 70
(1993) 612-615.%25

\snitem {\itel S. Ohsawa, T. Kawamura, N. Takamatsu and Y. Yusa}, Raleigh
scattering by aqueous colloidal silica as a cause for the blue
color of hydrothermal water, J. Volcanol. Geotherm. Res. 113
(2002) 49-60.%26

\snitem {\itel T. Quickenden and A. Hanlon,} The colours of water and ice, Chem.
Br. 36 (2000) 37-39; Chem. Br. 37 (2001) 18.%27

\snitem {\itel F. Bruge, M. Bernasconi.and M. Parrinello}, Ab initio simulation of
rotational dynamics of solvated ammonium ion in water, J. Am.
Chem. Soc. 121 (1999) 10883-10888.%28

\snitem {\itel L.  Pauling}, The Nature of the Chemical Bond, 2nd ed. (Cornell
University Press, New York., 1948).%29

\snitem {\itel R. A Mayanovic, A. J. Anderson, W. A. Bassett an I-M Chou},
Hydrogen bond breaking in aqueous solutions near the critical
point, Chem. Phys. Lett. 336 (2001) 212-218.%30

\snitem {\itel B. Ruscic, A. F. Wagner, L. B. Harding, R. L. Asher, D. Feller, D.
A. Dixon, K. A. Peterson, Y. Song, Q. M. Qian, C. Y. Ng, J. B. Liu
and W. W. Chen}, On the enthalpy of formation of hydroxyl radical
and gas-phase bond dissociation energies of water and hydroxyl, J.
Phys. Chem. A 106 (2002) 2727-2747.%31

\snitem {\itel S. J. Suresh and V. M. Naik,} Hydrogen bond thermodynamic
properties of water from dielectric constant data, J. Chem. Phys.
113 (2000) 9727-9732. %32

\snitem {\itel M. Henry,} Nonempirical quantification of molecular interactions in
supramolecular assemblies, ChemPhysChem 3 (2002) 561-569.%33

\snitem {\itel Y. Yamaguchi, N. Yasutake and M. Nagaoka,} Theoretical prediction
of proton chemical shift in supercritical water using gas-phase
approximation, Chem. Phys. Lett. 340 (2001) 129-136. %34

\snitem {\itel C. N. R. Rao}, Theory of hydrogen bonding in water, in F. Franks
(Ed), Water A comprehensive treatise, Vol. 1, Plenum Press, New
York, 1972) pp. 93-114.%35

\snitem {\itel E. Espinosa, E. Molins, C. Lecomte,} Hydrogen bond strengths
revealed by topological analyses of experimentally observed
electron densities, Chem. Phys. Lett. 285 (1998) 170-173. %36

\snitem {\itel A. Khan, }A liquid water model: Density variation from supercooled
to superheated states, prediction pf H-bonds, and temperature
limits. J. Phys. Chem. 104 (2000) 11268-11274.%37

\snitem {\itel A. Ranganathan, G. U. Kulkarni and C. N. R. Rao, }Understanding the
hydrogen bond in terms of the location of the bond critical point
and the geometry of the lone pairs, J. Phys. Chem. A 107 (2003)
6073-6081.%38

\snitem {\itel S. J. Grabowski, }A new measure of hydrogen bonding strength - ab
initio and atoms in molecules studies, Chem. Phys. Lett. 338
(2001) 361-366.%39

\snitem {\itel F. Bartha, O. Kapuy, C. Kozmutza and C. Van Alsenoy, }Analysis of
weakly bound structures: hydrogen bond and the electron density in
a water dimer, J. Mol. Struct. (Theochem) 666-667 (2003) 117-122. %40

\snitem {\itel R.Ludwig, }The effect of hydrogen bonding on the thermodynamic and
spectroscopic properties of molecular clusters and liquids, Phys.
Chem. Chem. Phys. 4 (2002) 5481-5487. %41

\snitem {\itel J. J. Dannenberg, }Cooperativity in hydrogen bonded aggregates.
Models for crystals and peptides, J. Mol. Struct. 615 (2002)
219-226.%42

\snitem {\itel T. Miyake and M. Aida, }Hydrogen bonding patterns in water
clusters: trimer, tetramer and pentamer, Internet Electron. J.
Mol. Des. 2 (2003) 24-32 \\<http://www.biochempress.com>.%43

\snitem {\itel W. A. P. Luck, }The importance of cooperativity for the properties
of liquid water, J. Mol. Struct. 448 (1998) 131-142. %44

\snitem {\itel E. Tombari, C. Ferrari and G. Salvetti, }Heat capacity anomaly in a
large sample of supercooled water, Chem. Phys. Lett. 300 (1999)
749-751.%45

\snitem {\itel M. I. Heggie, C. D. Latham, S. C. P. Maynard and R. Jones,}
Cooperative polarisation in ice Ih and the unusual strength of the
hydrogen bond, Chem. Phys. Lett. 249 (1996) 485-490. [Back] L.%46

\snitem {\itel F. N. Keutsch and R. J. Saykally, }Water clusters: Untangling the
mysteries of the liquid, one molecule at a time, PNAS 98 (2001)
10533-10540.%47

\snitem {\itel J. Higo, M. Sasai, H. Shirai, H. Nakamura and T. Kugimiya, }Large
vortex-like structures of dipole field in computer models of
liquid water and dipole-bridge between biomolecules, Proc. Natl.
Acad. Sci. USA 98 (2001) 5961-5964.%48

\snitem {\itel S. Woutersen and H. J. Bakker, }Resonant intermolecular transfer of
vibrational energy in liquid water, Nature 402 (1999) 507-509.%49

\snitem {\itel D. P. Shelton, }Collective molecular rotation in water and other
simple liquids, Chem. Phys. Lett. 325 (2000) 513-516.%50

\snitem {\itel N. Yoshii, S. Miura and S. Okazaki, }A molecular dynamics study of
dielectric constant of water from ambient to sub-and supercritical
conditions using a fluctuating-charge potential model, Chem. Phys.
Lett. 345 (2001) 195-200.%51

\snitem {\itel L. A.Guildner, D. P. Johnson, and F. E. Jones, }Vapor pressure of
water at its triple point, J. Res. Natl. Bur. Stand. 80A (1976)
505-521. %52

\snitem {\itel P. W. Bridgman, }Water, in the liquid and five solid forms, under
pressure, Proc. Am. Acad. Arts Sci. 47 (1912) 439-558.%53

\snitem {\itel L. Mercury, P. Vieillard and Y. Tardy, }Thermodynamics of ice
polymorphs and `ice-like' water in hydrates and hydroxides, Appl.
Geochem. 16 (2001) 161-181.%54

\snitem {\itel M. Song, H. Yamawaki, H. Fujihisa, M. Sakashita and K. Aoki,}
Infrared investigation on ice VIII and the phase diagram of dense
ices, Phys. Rev. B 68 (2003) 014106.%55

\snitem {\itel B. Schwager, L. Chudinovskikh, A. Gavriliuk and R. Boehler,}
Melting curve of H$_2$O to 90 GPa measured in a laser-heated
diamond cell, J. Phys: Condens. Matter 16 (2004) S1177-S1179. %56

\snitem {\itel H. Ohtaki, }Effects of temperature and pressure on hydrogen bonds
in water and in formamide, J. Mol. Liquids 103-104 (2003) 3-13. %57

\snitem {\itel O. Mishima and H. E. Stanley, }The relationship between liquid,
supercooled and glassy water, Nature 396 (1998) 329-335. %58

\snitem {\itel G. C. Leon, S. Rodriguez Romo and V. Tchijov, }Thermodynamics of
high-pressure ice polymorphs: ice II, J. Phys. Chem. Solids 63
(2002) 843-851.%59

\snitem {\itel W. A. P. Luck, }in Water and Ions in Biological Systems, eds. A.
Pullman, V. Vasileui and L. Packer (Plenum: New York, 1985) p. 95.%60

\snitem {\itel G. H. Pollack, }Is the cell a gel-and why does it matter? Jap. J.
Physiol. 51 (2001) 649-660.%61

\snitem {\itel F. Franks, }Introduction - water, the unique chemical, in F. Franks
(Ed), Water A comprehensive treatise, Vol. 1, (Plenum Press, New
York, 1972) pp. 1-20.%62

\snitem {\itel D. Auerbach, }Supercooling and the Mpemba effect; when hot water
freezes quicker than cold, Am. J. Phys. 63 (1995) 882-885.%63

\snitem {\itel P. M. Wiggins, }High and low-density water in gels, Prog. Polymer.
Sci. 20 (1995) 1121-1163.%64

\snitem {\itel A. V. Kondrachuk, V. V. Krasnoholovets, A. I. Ovcharenko and E. D.
Chesnokov, }Determination of the water structuring by the pulsed
NMR method, Khim. Fiz. 12 (1993) 1006-1010; translated in Sov.
Jnl. Chem. Phys. 12 (1994) 1485-1492. %65

\snitem {\itel A. F. Heneghan and A. D. J. Haymet, }Liquid-to-crystal
heterogeneous nucleation: bubble accelerated nucleation of pure
supercooled water, Chem. Phys. Lett. 368 (2003) 177-182. %66

\snitem {\itel C. H. Cho, J. Urquidi and G. Wilse Robinson, }Molecular-level
description of temperature and pressure effects on the viscosity
of water, J. Chem. Phys. 111 (1999) 10171-10176. %67

\snitem {\itel C. H. Cho, J. Urquidi, S,. Singh and G. Wilse Robinson, }Thermal
offset viscosities of liquid H2O, D2O, and T2O, J. Phys. Chem. B,
103 (1999) 1991-1994.%68

\snitem {\itel G. S. Kell, }J. Chem. Eng. Data 20(1) (1975) 97.%69

\snitem {\itel D. R. Lide Ed., }CRC Handbook of chemistry and physics, 80th Ed.
(CRC Press, Boca Raton, 1999). %70

\snitem National Institute of Standards and Technology, A gateway to the
data collections. Available at http://webbook.nist.gov (accessed
19 January 2001). %71

\snitem {\itel G. P. Johari, A. Hallbrucker and E. Mayer, }Two calorimetrically
distinct states of liquid water below 150 Kelvin, Science 273
(1996) 90-92.%72

\snitem {\itel M-L. Tan, J. T. Fischer, A. Chandra, B. R. Brooks and T. Ichiye, }A
temperature of maximum density in soft sticky dipole water, Chem.
Phys. Lett. 376 (2003) 646-652.%73

\snitem {\itel K. Kiyohara, K. E. Gubbins and A. Z. Panagiotopoulos, }Phase
coexistence properties of polarizable water models, Molecular
Phys. 94 (1998) 803-808.%74

\snitem {\itel D. van der Spoel, P. J. van Maaren and H. J. C. Berendsen, }A
systematic study of water models for molecular simulation:
Derivation of water models optimized for use with a reaction
field, J. Chem. Phys. 108 (1998) 10220-10230. %75

\snitem {\itel M. W. Mahoney and W. L. Jorgensen, }Diffusion constant of the TIP5P
model of liquid water, J. Chem. Phys. 114 (2001) 363-366. %76

\snitem {\itel P. G. Kusalik and I. M. Svishchev, }The spatial structure in liquid
water, Science 265 (1994) 1219-1221.%77

\snitem {\itel L. A. Baez and P. Clancy, }Existence of a density maximum in
extended simple point-charge water, J. Chem. Phys. 101 (1994)
9837-9840.%78

\snitem {\itel I. M. Svishchev, P. G. Kusalik, J. Wang and R. J. Boyd,}
Polarizable point-charge model for water. Results under normal and
extreme conditions, J. Chem. Phys. 105 (1996) 4742-4750.%79

\snitem {\itel M. W. Mahoney and W. L. Jorgensen, }A five-site model for liquid
water and the reproduction of the density anomaly by rigid,
nonpolarizable potential functions, J. Chem. Phys. 112 (2000)
8910-8922.%80

\snitem {\itel S. W. Rick, }Simulation of ice and liquid water over a range of
temperatures using the fluctuating charge model, J. Chem. Phys.
114 (2001) 2276-2283.%81

\snitem {\itel P. J. van Maaren and D. van der Spoel, }Molecular dynamics of water
with novel shell-model potentials, J. Phys. Chem. B 105 (2001)
2618-2626.%82

\snitem {\itel H. A. Stern, F. Rittner, B. J. Berne and R. A. Friesner, }Combined
fluctuating charge and polarizable dipole models: Application to a
five-site water potential function, J. Chem. Phys. 115 (2001)
2237-2251. %83

\snitem {\itel A. A. Kornyshev, A. M. Kuznetsov, E. Spohr and J. Ulstrup,}
Kinetics of proton transport in water, J. Phys. Chem. B 107 (2003)
3351-3388.%84

\snitem {\itel A. V. Gubskaya and P. G. Kusalik, }The total molecular dipole
moment for liquid water, J. Chem. Phys. 117 (2002) 5290-5302. %85

\snitem {\itel J. N. Murrell, A. D. Jenkins, }Properties of Liquids and solutions,
2nd Ed. (John Wiley and Sons, Chichester, England, 1994).%86

\snitem {\itel A. Wallqvist and R. D. Mountain, }Molecular models of water:
Derivation and description,  Reviews in Computational Chemistry13
(1999) 183-247. %87

\snitem {\itel A. G. Kalinichev, }Molecular simulations of liquid and
supercritical water: Thermodynamics, structure, and hydrogen
bonding. In: Molecular Modeling Theory: Applications in the
Geosciences. Ed. R. T. Cygan and J. D. Kubicki. Rev. Mineralogy
Geochem.42 (2001) 83-129.%88

\snitem {\itel B. Guillot, }A reappraisal of what we have learnt during three
decades of computer simulations on water, J. Mol. Liquids 101
(2002) 219-260.%89

\snitem {\itel J. L. Finney, }The water molecule and its interactions: the
interaction between theory, modelling and experiment, J. Mol.
Liquids, 90 (2001) 303-312. %90

\snitem {\itel S. W. Rick, }A reoptimization of the five-site water potential
(TIP5P) for use with Ewald sums, J. Chem. Phys. 120 (2004)
6085-6093.%91

\snitem {\itel H. W. Horn, W. C. Swope, J. W. Pitera, J. D. Madura, T. J. Dick,
G. L. Hura, T. Head-Gordon, }Development of an improved four-site
water model for biomolecular simulations: TIP4P-Ew, J. Chem. Phys.
120 (2004) 9665-9678. %92

\snitem {\itel D. van der Spoel, P. J. van Maaren and H. J. C. Berendsen, }A
systematic study of water models for molecular simulation:
Derivation of water models optimized for use with a reaction
field, J. Chem. Phys. 108 (1998) 10220-10230.%93

\snitem {\itel A. Brodsky, }Is there predictive value in water computer
simulations? Chem. Phys. Lett. 261 (1996) 563-568. %94

\snitem {\itel B.Guillot and Y. Guissani, }How to build a better pair potential
for water, J. Chem. Phys. 114 (2001) 6720-6733.%95

\snitem {\itel E. Sanz, C. Vega, J. L. F. Abascal and L. G. MacDowell, }Tracing
the phase diagram of the four-site water potential (TIP4P). J.
Chem. Phys. 121 (2004) 1165-1166. E. Sanz, C. Vega, J. L. F.
Abascal and L. G. MacDowell, Phase diagram of water from computer
simulation, Phys. Rev. Lett. 92 (2004) 255701.%96

\snitem {\itel P. Jedlovszky and J. Richardi, }Comparison of different water
models from ambient to supercritical conditions: A Monte Carlo
simulation and molecular Ornstein-Zernike, J. Chem. Phys. 110
(1999) 8019-8031.%97

\snitem {\itel J. M. Sorenson, G. Hura, R. M. Glaeser and T. Head-Gordon, }What
can x-ray scattering tell us about the radial distribution
functions of water? J. Chem. Phys. 113 (2000) 9149-9161. %98

\snitem {\itel V. E. Petrenko, M. L. Dubova, Y. M. Kessler and M. Y. Perova,}
Water in computer experiment: Contradiction in parameterization of
potentials, Russ. J. Phys. Chem. 74 (2000) 1777-1781. %99

\snitem {\itel Ph. Wernet, D. Nordlund, U. Bergmann, M. Cavalleri, M. Odelius, H.
Ogasawara, N. A. Naslund, T. K. Hirsch, L. Ojamae, P. Glatzel, L.
G. M. Pettersson and A. Nilsson,} The structure of the first
coordination shell in liquid water, Sciencexpress 1 April 2004;
10.1126/science.1096205 ; Science 304 (2004) 995-999.%100

\snitem {\itel J. C. Dore, M. A. M. Sufi and M. -C. Bellissent-Funel, }Structural
change in D2O water as a function of temperature; the isochoric
temperature derivative function for neutron diffraction, Phys.
Chem. Chem. Phys. 2 (2000) 1599-1602. %101

\snitem {\itel F. Corzana, M. S. Motawia, C. Herve du Penhoat, S. Perez, S. M.
Tschampel, R. J. Woods and S. B. Engelsen, }A hydration study of
(1$\rightarrow$4) and (1$\rightarrow$6) linked a-glucans by comparative 10 ns molecular
dynamics simulations and 500-MHz NMR, J. Comput. Chem. 25 (2004)
573-586.%102

\snitem {\itel G. A. Martynov, }Structure of fluids from the statistical mechanics
point of view, J. Mol. Liquids 106 (2003) 123-130.%103

\snitem {\itel A. Eisenstein and N. S. Gingrich, }Phys. Rev. 62 (1942) 261.%104

\snitem {\itel K. Coutinho, R. C. Guedes, B. J. C. Cabral and S. Canuto,}
Electronic polarization of liquid water: converged Monte
Carlo-quantum mechanics results for the multipole moments, Chem.
Phys. Lett. 369 (2003) 345-353.%105

\snitem {\itel J. Higo, M. Sasai, H. Shirai, H. Nakamura and T. Kugimiya,} Large
vortex-like structures of dipole field in computer models of
liquid water and dipole-bridge between biomolecules, Proc. Natl.
Acad. Sci. USA 98 (2001) 5961-5964.%106

\snitem {\itel S. M. Pershin, }OH-Group vibration spectrum of metastable
hydrogen-bound states of liquid water, Phys. Wave Phen. 11 (2003)
89-95.%107

\snitem {\itel B. Vyb$\acute{}$iral and P. Vor$\acute{}$acek, }\lk
Autothixotropy\pk of water \lt an unknown physical phenomenon,
arXiv.org Physics
e-Print archive physics/0307046 (2003).%108

\snitem {\itel M. F. Chaplin, }A proposal for the structuring of water, Biophys.
Chem. 83 (2000) 211-221.%109

\snitem {\itel A. Muller, H. Bogge and E. Diemann, }Structure of a
cavity-encapsulated nanodrop of water, Inorg. Chem. Commun. 6
(2003) 52-53; Corrigendum: A. Muller, H. Bogge and E. Diemann,
Inorg. Chem. Commun. 6 (2003) 329. %110

\snitem {\itel A. H. Narten, M. D. Danford and H. A. Levy, }X-Ray diffraction
study of liquid water in the temperature range 4-200�C, Faraday
Discuss. 43 (1967) 97-107.%111

\snitem {\itel M. R. Battaglia, A. D. Buckingam, and J. H. Williams, }Chem. Phys. Lett. 78, 421 (1981).%112

\snitem {\itel L. J. Lowden and D. Chandler, }J. Chem. Phys. 61, 5228 (1974).%113

\snitem {\itel L. J. Lowden and D. Chandler, }J. Chem. Phys. 59, 6587 (1974).%114

\snitem {\itel Shi Xiangian and L. S. Bartell, }J. Phys. Chem. 92, 5667
(1988). %115

\snitem {\itel G. Garkstrom, P. Linse, A. Wallquist, and B. Johnson, }J. am. Chem.
Soc. 105, 3777 (1983).%116

\snitem {\itel N. B. Rozhdestvenskaya and L. V. Smirnova. }JETP Lett. 44, 165
(1986).%117

\snitem {\itel N. B. Rozhdestvenskaya, L. V. Smirnova, }J. Chem. Phys. 195,
1223.(1991).%118

\snitem {\itel P. Linse, }J. Am. Chem. Soc. 106, 5425 (1984).%119

\snitem {\itel M. Claessens, M. Ferrario, and J. P. Rychaert, }Mol. Phys. 50, 217
(1983).%120

\snitem {\itel J. S. Hoye and G. Stell, }J. Chem. Phys. 66, 795 (1977).%121

\snitem {\itel W. H. Martin and S. H. Lehrman, }J. Phys. Chem. 26, 75 (1922).%122

\snitem {\itel L. Letamendia, M. Belkadi, O. Eloutassi, J. Rouch,
D. Risso,
P. Cordero, A. Z. Patashinski, }Physica A 354, p. 34-48 (2005).%123

\snitem {\itel W. H. Flygare, }Molecular Structure and Dynamics
(Prentice-Hall,
New Jersey, 1978).%124

\snitem {\itel L. Letamendia, M. Belkadi, O. Eloutassi, G. Nouchi,
C. Vaucamps, S. Iakovlev, N. Rozhdestvenkaya, L. V. Smirnova, V.
Runova, }Phys.
Rev. E 48 5327 (1996).%125

\snitem {\itel L. Letamendia, M. Belkadi, O. Eloutassi, E.
Pru-Lestret, G. Nouchi, J. Rouch, D. Blaudez, F. Mallamace, N.
Micali, C. Vasi,}
Phys. Rev. E 54 5327 (1996).%126

\snitem {\itel M. Grimsditch, N. River, }Appl. Phys. Lett. 58 2345 (1991).%127

\snitem {\itel A. Z. Patashinski, M. A. Ratner, }J. Chem. Phys.
120 2814
(2004).%128

\snitem {\itel R. T. Morrison, R. N. Boyd, }Organic Chemistry,
Prentice-Hill,
Englewood Cliffs, NJ, 1992.%129

\snitem {\itel D. Eisenberg, W. Kauzman, }The Structure and
Properties of
Water, University Press, New York, Oxford, 1969.%130

\snitem {\itel N. B. Rozhdestvenskaya, K. Eidner,} Vestn. LGU,
Ser. Phys.
Chem. 10, 50 (1978) (in Russian).%131

\snitem {\itel N. B. Rozhdestvenskaya and E. N. Gorbachova,
}Optics Commun.
30, 383 (1979).%132









%\snitem {\itel


}
